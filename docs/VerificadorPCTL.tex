\documentclass[12pt,letterpaper,spanish]{article}
\usepackage{babel}
\usepackage[spanish]{layout}
\usepackage[utf8]{inputenc}
\usepackage{listings}
\usepackage{verbatim}

\setlength{\unitlength}{1cm}
\setlength{\parindent}{0cm}

\pagestyle{plain}

\author{Laura Alicia Leonides Jiménez}
\title{Verificador para \textsc{pctl}}

\date{ } 

\begin{document}
\maketitle

En el presente documento se especifica cómo fue implementado el programa, cuáles archivos
se entregan en la distribución anexa a la versión electrónica del documento y cómo debe
ejecutarse el programa.
\\\\
Asimismo, se incluyen algunos ejemplos de los archivos en los que debe especificarse el modelo
y la fórmula a verificar.

\section{Implementación}
El programa fue realizado en Haskell 98 y contiene los siguientes módulos:
\begin{description}
 \item [SintaxisPCTL.] En él se define la sintaxis de las
fórmulas de \textsc{Pctl} que se podrán procesar.
 \item [VerificadorPCTL.] Se encarga de verificar un modelo y una fórmula.
 \item [Rela.] Contiene funciones auxiliares para manejar la relación de transición  con 
probabilidades.
 \item [Modelo.] Contiene las definiciones de tipos de datos del modelo.
 \item [Main.] Módulo principal del sistema. Encargado de obtener el archivo
con las definiciones y de procesar la información que ahí se encuentra (modelo y fórmula)
necesarias para especificar un modelo.
 \item [Lector.] Se encarga de leer el archivo en el que se encuentra
la definición del modelo y la fórmula. Parsea cada una de las partes involucradas.
 \item [ArrayUtil.] Contiene funciones auxiliares para facilitar el manejo de arreglos
y matrices.
\end{description}

Asimismo, se utilizó la biblioteca Haskell \textsc{dsp} como auxiliar en el manejo de matrices.

Todos los archivos fuente utilizan la codificación \textsc{utf}-8.

\section{Distribución}
Junto con la versión electrónica de este documento, se entrega el archivo \texttt{VerificadorPCTL.tar.gz}.
Dicho archivo debe descomprimirse y entonces se tendrá la siguiente estructura de directorios:
\begin{verbatim}
VerificadorPCTL
|-- prueba.dat
`-- src
    |-- ArrayUtil.hs
    |-- Lector.hs
    |-- Main.hs
    |-- Matrix
    |   |-- LU.hs
    |   `-- Matrix.hs
    |-- Modelo.hs
    |-- Rela.hs
    |-- SintaxisPCTL.hs
    `-- VerificadorPCTL.hs

\end{verbatim}


\section{Ejecución}

\textbf{Con un intérprete}
\\\\
Basta con cargar el módulo \texttt{Main} en un intérprete de Haskell (las
pruebas se realizaron utilizando \textsc{ghc}i Versión 6.8.2) e invocar a la función
\texttt{main}, de la siguiente manera:
\begin{verbatim}
[cotupha@ronja src]$ ghci Main
GHCi, version 6.8.2: http://www.haskell.org/ghc/  :? for help
Loading package base ... linking ... done.
[1 of 9] Compiling SintaxisPCTL     ( SintaxisPCTL.hs, interpreted )
[2 of 9] Compiling Matrix.Matrix    ( Matrix/Matrix.hs, interpreted )
[3 of 9] Compiling Matrix.LU        ( Matrix/LU.hs, interpreted )
[4 of 9] Compiling Modelo           ( Modelo.hs, interpreted )
[5 of 9] Compiling Lector           ( Lector.hs, interpreted )
[6 of 9] Compiling Rela             ( Rela.hs, interpreted )
[7 of 9] Compiling ArrayUtil        ( ArrayUtil.hs, interpreted )
[8 of 9] Compiling VerificadorPCTL  ( VerificadorPCTL.hs, interpreted )
[9 of 9] Compiling Main             ( Main.hs, interpreted )
Ok, modules loaded: VerificadorPCTL, Main, ArrayUtil, Modelo, Matrix.LU,
Matrix.Matrix, Lector, SintaxisPCTL, Rela.
*Main>  
\end{verbatim}
Después de especificar el nombre del archivo, el programa parsea dicho archivo y muestra en pantalla
los estados que satisfacen a la fórmula dada en el modelo especificado, como se muestra a continuación:
\begin{verbatim}
Dame el nombre del archivo con la especificaciónn del modelo M y la
fórmula a verificar
../prueba.dat
Los estados s tales que M,s|=a son:
fromList [2,3,4]
\end{verbatim}

\section{Definición del modelo y fórmula}
Es necesario proporcionar al programa un archivo en el que se encuentren las definiciones necesarias
para determinar el modelo y la fórmula a verificar.
Deben especificarse, en el orden presentado, los siguientes componentes:
\begin{itemize}
 \item Conjunto de átomos.
 \item Número de estados.
 \item Relación de transición con probabilidades.
 \item Función de etiquetamiento.
 \item Fórmula.
\end{itemize}
A continuación se incluye un ejemplo de dicho archivo:
\begin{lstlisting} [mathescape,basicstyle=\small,columns=flexible,xleftmargin=0.9cm,numbers=left,label=ejemplo,language=C++,extendedchars=true]
{not, try, fail, succ}
4
(2, 1.0)
(2, 0.01),(3, 0.01),(4, 0.98)
(1, 1.0 )
(4, 1.0 )
#1>{not}#2>{try}#3>{fail}#4>{succ}
(P >= 0.9 [X (Or (Not(Lit try))(Lit succ))])
\end{lstlisting}
Ejemplos de otras fórmulas:
\begin{lstlisting} [mathescape,basicstyle=\small,columns=flexible,xleftmargin=0.9cm,numbers=left,label=ejemploF,language=C++,extendedchars=true]
************ X **************
(P >= 0.9 [X (Or (Not(Lit try))(Lit succ))])
(P >= 0.9 [X (Not (And (Lit try)(Not (Lit succ))))])
(P <= 0.99 [X (Not (And (Lit try)(Not (Lit succ))))])
(P < 0.99 [X (Not (And (Lit try)(Not (Lit succ))))])
************ Uk *************
(P > 0.98 [Uk 2 (Top) (Lit succ)])
(P >= 1.0 [Uk 2 (Top) (Lit succ)])
************ U **************
(P > 0.99 [U (Lit try) (Lit succ)])
(P < 0.99 [U (Lit try) (Lit succ)])
************ Props **********
(Not (And (Lit try)(Not (Lit succ))))
(Lit succ)
(Top)
(Not (Lit try))
(And (Lit try) (Lit fail))
(Lit try)
\end{lstlisting}

\end{document}